\documentclass[10pt]{article}

\usepackage{palatino}
\linespread{1.05}

%\pagestyle{empty}

\author{Adam Lloyd}
\title{The Silent Partner}
\date{17 October 2010}

\begin{document}
\maketitle

``It's here,'' said the eyes.

``Awesome,'' replied the mind.  ``Imagination, wake up!  Fingers, at
the ready!  Eyes, go ahead!''

``Exercise 59,'' eyes reported, ``Silent Partner.''

Mind gave the phrase a few microseconds to bounce around its practically
infinite data banks.  Satisfied that nothing important was found, it
allowed the eyes to continue.

``Contemplate for awhile, then slowly write down a conversation between
several people in which one person says very little, or nothing.''

Eyes paused for a moment, in case imagination was already on top of
things and had something to blurt out.  It wasn't, and it didn't.  Eyes
continued.

``Make this silent partner a crucial part of the situation, and don't
ignore this character just because they are not speaking.''

Mind was a little indignant.  ``I've had to try to comprehend a
ridiculously large-scale notion of this `God' thing that people are
always going on about and then try to regurgitate it in 200 elegant
words.  I've had to create a religion from the eyes of a bunch of silly
little things that don't even \emph{have} eyes and 'worship'---as if
they're even capable of that!---some bigger silly thing with no will of
its own! And now!''  Mind paused for effect.  ``Now I'm supposed to
write a \emph{conversation}?  That's \emph{it}?  Sounds boring.''

Mind thought it had made it clear that this was imagination's cue.
Fingers thought so, too; they were poised to begin striking out what
they were sure would be a very inspired sequence of letters.  (And you
can't judge them, of course.  \emph{Any} sequence of letters seems very
inspired to fingers.)

Mind did its best impression of a sigh.  Imagination always \emph{had}
been a little dull.  ``Er-hem,'' mind said, hoping to sound convincing
enough to keep everyone from remembering it didn't have a throat to
clear.  It did its best to sound jovial.  ``Imagination!  What've you
got for me?''

``Oh,'' said the imagination, ``nothing.''

The mind mumbled something about the imagination being terrible at its
job nowadays.  ``Alright, eyes, give it to us again.  Listen up this
time, imagination.''

``Right-o!  We've got `Exercise 59, Silent Partner'.''

``You gettin' this, imagination?'' the mind interruped.

The imagination grunted.

Eyes continued.  ``Contemplate for awhile, then slowly write down a
conversation between several people in which one person says very
little, or nothing.''

The mind, in a good mood for ranting, interrupted again here: ``Slowly?
Why slowly?  Hey, fingers, you can write as fast as you want, okay?''
The eyes politely paused to give the mind a moment to get back on track.

``Make this silent partner a crucial part of the situation, and don't
ignore this character just because they are not speaking.''

``Yep, conversation,'' the mind confirmed. ``Super hard.  Alright,
imagination, got it this time?''

The imagination gave another noncommittal grunt.  The mind noticed in
annoyance that the imagination didn't have a throat with which to give
noncommittal grunts.  It would have grumpily pointed that out if the wit
hadn't chosen that moment to jump in.

``Hey, guys! Guys! I got this! We don't even need imagination at all.''
Wit considered flashing a grin around, but decided that its lack of a
mouth made it more trouble than it was worth.  ``You can just write up
\emph{any conversation you've ever had}.  That works, doesn't it, quiet
guy?''

``Yeah. Funny,'' the mind icily replied. ``Shut up, wit. Go back to
your usual state of slow uselessness.''

Silence.

``Oh, come on, imagination!'' The mind was exasperated. ``You've really
got nothing?''

``Well, I guess\ldots maybe the quiet one could be a computer.''

``Dumb,'' said the mind.  ``We use computers for everything.  Besides,
you'd \emph{expect} a computer to be quiet.  Give me something more
interesting.''

Silence.

``Argh.  I hate my job sometimes.''  The mind shut itself off.

\vskip 1cm

``Alright, imagination, you got anything yet?''

No answer.

``Alright, let's break it down.  You don't need to give me a whole
story at once.  Let's just come up with a situation.''

The fingers tapped idly, in silence.

``Imagination,'' mind yelled, ``that's your cue!''

The reply came quietly.  ``Well\ldots I guess maybe someone could be
concentrating on something.''

``Sure,'' said the mind. ``It's not the best of ideas, but we'll work
with it.  Concentrating on what?''

``Dunno.''

The mind did its impression of a sigh again.  ``Maybe we should take
another look at the assignment.  Eyes!''

``Gotcha, buddy! Sure thing!''

The mind stopped the eyes to ask why they were so darned enthusiastic
about this whole affair.  They mentioned something about cute little
smilies on the writing assignment.  Mind told them to roll themselves
and go back to reading.

``Alright, contemplate for awhile.''

``Check,'' the mind interjected sarcastically.

``Slowly write down a conversation between several people in which one
person says very little, or nothing.  Make this silent partner a crucial
part of the situation, and don't ignore this character just because they
are not speaking.''

Mind realized that it didn't actually need to hear this from the eyes;
it had already heard it enough to have it committed to memory.
Nevertheless, hearing it again might help the still-silent imagination.

``Oh, and it says it should be at least 700 words.''

``Oh, boy,'' said the mind. ``That's a lot.  Especially at this rate.''

Imagination didn't take the hint.

``So! Imagination!''

Silence.

The mind was beginning to wish dearly that it could hire a new
imagination; this one always failed to deliver.

``Al\emph{right}!'' The mind decided to keep pushing. ``So we've got
someone concentrating on something.  On what?''

``Something\ldots important,'' the imagination offered.

The mind told the eyes to roll again. ``Helpful. Anything more
specific?''

If the imagination worked by turning gears, the mind would have been
able to hear the sound of the most arduous gear-turning ever undertaken.
Of course, the imagination \emph{didn't} work by turning gears (and
clearly wouldn't even have been able to pretend that it did), so the
mind was instead left in silence with the impression that the
imagination was working very hard.

Imagination finally spoke up. ``Math problems?''

The eyes rolled themselves even before they were told.
``Sounds riveting,'' said the mind.

The imagination tried again. ``A computer program?''

``Not much better.''

The imagination spent a few more seconds giving off that impression of
being hard at work. ``Books!''

``Let me think,'' the mind said, requiring no time to think. ``No.''

``Hackers!''

The mind acknowledged that this was probably the best idea yet.  ``Too
technical to be done well, though.  Anything else?''

The imagination paused thoughtfully.  ``A \emph{bomb}!''

``Alright,'' said the mind. ``\emph{Clearly}, the imagination's got
nothing.''

The fingers stretched themselves.  They were ready to pass this message
on, maybe in exchange for a new assignment that the imagination could
handle.

The mind halted them. ``Let's give it another day.  Maybe the
imagination will come up with something.''

The imagination grunted.

``Wait!'' said the mind.  It told the eyes to widen. ``What's that you
just said?''

``Who, me?'' The imagination was confused. ``Nothing.''

``Exactly.'' The mind ordered the mouth to grin. ``Hey, eyes!  Let's take
it from the top.  Fingers, listen up!  Let's go!''

\end{document}
